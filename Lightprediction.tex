\documentclass{article}
\usepackage{amsmath, amssymb}

\begin{document}

\section*{Formal Definition}

The survival probability (prediction window) is given by:
\[
P_C = \sigma\big( \alpha p + \beta S - \gamma D + \eta R_{\text{net}} + \delta u \big),
\]
with survival/prediction iff
\[
P_C \geq P^*.
\]

Where:
- $r =$ restraint  
- $a =$ alignment  
- $p =$ persistence  
- $\theta =$ stress  
- $u =$ uncertainty  

The predicted arrival time of light from $A$ to $C$ along path $\Gamma$ is:
\[
t_{AC}(\varepsilon, \kappa) = \int_{\Gamma} \frac{ds}{c \,\big(1 - \kappa \cdot n(s)\big)} + \text{GR} + \text{(kinematic terms)}.
\]

Variance of prediction error shrinks as:
\[
\alpha p + \beta S - \gamma D + \eta R_{\text{net}} + \delta u \quad \text{grows.}
\]

\section*{Operational Rule (First Signal Synchronization)}

1. Source restrains (emission discipline).  
2. Choir aligns (two-way clock sync across network).  
3. Pick the gauge $(\varepsilon, \kappa)$ that maximizes $P_C$.  
4. Set release rate $m^*$ so that $R \geq 1$ under observed noise.  
5. Predict arrivals: report
\[
t_{AC} \pm z_q \, \sqrt{\mathrm{Var}[\Delta t]}.
\]
6. Adapt release as stress $\theta$ changes.  

\end{document}
